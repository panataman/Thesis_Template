\section*{Studiere DAS} \todo{dont forget to delete!}
Die Einleitung ist der wichtigste Textabschnitt einer Hochschularbeit – und zwar
gleichgültig, ob es sich um eine Diplomarbeit, eine Bachelorthesis oder eine
Staatsexamensarbeit handelt. Hier wird der Leser ins Thema eingeführt, hier wird
die Fragestellung formuliert, und hier werden die methodischen
Grundsatzentscheidungen getroffen.

Tipp: Man sollte die Einleitung, auch wenn das sogar von Prüferseite geäußert
wird, nicht zum Schluss schreiben, sondern sie vorab verfassen und sie während
der Niederschrift als Arbeitsinstrument verwenden. Denn hieran – und nur hieran
– zeigt sich, ob das Untersuchungsvorhaben klappt. Häufig zeigt sich auch,
weshalb es nicht so gut klappt.

All das weiß auch der Prüfer. Daher steht im Grunde die Note schon nach der
Lektüre der Einleitung fest – zumindest wird es schwierig sein, den Prüfer davon
zu überzeugen, dass es sich doch um eine gute Arbeit handelt, wenn die
Einleitung misslungen ist.  Aber wie schreibt man nun eine gute Einleitung? Im
Folgenden einige grundlegende Hinweise:

Hilfreich ist eine Aufteilung in vier Teile, die sinnvoll aufeinander aufbauen.
Das heißt nicht, dass es nicht auch andere Möglichkeiten gibt, aber so
funktioniert es mit Sicherheit – bei jedem Thema.  Das vorgeschlagene Schema
sieht wie folgt aus:

\begin{enumerate}
	\item Hinführung (2 Absätze); diese besteht im Idealfall aus dem
		\begin{enumerate}
			\item Einstieg (etwas, was an die allgemeine Erfahrung
			anknüpft und unmittelbar ersichtlich ist) und
			\item einem weiteren Absatz, in dem – ausgehend vom
			Einstieg – auf das eigentliche Thema fokussiert wird.
		\end{enumerate}
		In einer Arbeit über Online-Marketing mit Facebook
		beispielsweise würde es im 1. Absatz um Online-Marketing
		allgemein gehen und im 2. Absatz auf die besonderen
		Anforderungen im Zusammenhang mit Facebook verwiesen.
		(Kontrollfrage: „Worum geht es hier?“)

	\item Fragestellung (1 Satz), auch als Problemstellung oder
	Forschungsfrage bezeichnet. Dabei handelt es sich im Idealfall
	tatsächlich nur um einen Satz oder einen Fragesatz. Die Fragestellung
	muss sich organisch aus dem 2. Absatz der Hinführung
	ableiten lassen. Im Idealfall ergibt sich für den Leser selbst
	angesichts des bisher Vermittelten an exakt dieser Stelle eine
	Frage, die er
	dann vom Verfasser bzw. der Verfasserin
	formuliert bekommt. Damit einher geht die Vorgabe, dass die
	Fragestellung wirklich nur exakt eine einzige Frage bzw. ein
	Forschungsproblem betrifft, nicht zwei oder drei. Wenn das passiert, hat
	man schon an dieser Stelle etwas falsch gemacht.  (Kontrollfrage: „Was
	ist das
		Problem?“)
	\item Operationalisierung der Fragestellung (3 bis 4 Sätze); dabei wird die Frage beantwortet, welche inhaltlichen und methodischen Aspekte im
		Zusammenhang mit der Klärung der Forschungsfrage wichtig sind. Dabei dürfen dann durchaus weiterführende Fragen gestellt oder Hypothesen
		formuliert werden, die sich aber wiederum
		auf die zentrale Fragestellung zurückführen lassen müssen. (Kontrollfragen: „Welche Überlegungen sind damit verknüpft? Was brauche ich, um
		die Forschungsfrage zu beantworten?“)
	\item Untersuchungsverlauf (pro Kapitel ein – kurzer – Absatz mit Verweis auf die Kapitelnummer); hier wird geklärt, was Inhalt der einzelnen
		Kapitel ist. Neue inhaltliche oder methodische Aspekte sollten hier nicht mehr vorkommen, hier geht es nur noch um das Wie, nicht mehr um
		das Was. (Kontrollfrage: „Wie wird
		vorgegangen?“)
\end{enumerate}

Hinweis: Die Schritte 3 und 4 werden häufig vermischt, oder die Operationalisierung unterbleibt ganz. Wir möchten nochmals deutlich
darauf hinweisen, dass die inhaltlichen, vor allem aber die methodischen und strukturellen Überlegungen in Bezug auf die
Fragestellung ein unverzichtbares Instrument sind, um die grundlegenden Zusammenhänge der Untersuchung zu klären. Demgegenüber
dient der Untersuchungsverlauf lediglich dazu, die Überlegungen im Hinblick auf die Gliederung in eine sinnvolle Abfolge zu
bringen.

Wie gesagt, dieses Schema funktioniert mit jedem Thema – wenn nicht, deutet das eher darauf hin, dass mit dem Thema etwas nicht
stimmt. Konkrete Fragen dazu beantworten wir gern im o:T-Forum. Ansonsten lässt sich auch im Rahmen einer Kurzberatung klären, ob
die Einleitung schlüssig ist. Dazu ist keine Beauftragung eines Lektorats notwendig.
