\chapter{Einleitung}
\label{chap:einleitung}

%%%%%%%%%%%%%%%%%%%%%%%%%%%%%%%%%%%%%%%%%%%%%%%%%%%%%%%
%%%%%%%%%%%%%%%%%%%%%%%%%%%%%%%%%%%%%%%%%%%%%%%%%%%%%%%
%%%%%%%%%%%%%%%%%%%%%%%%%%%%%%%%%%%%%%%%%%%%%%%%%%%%%%%
%\section*{Studiere DAS} \todo{dont forget to delete!}
Die Einleitung ist der wichtigste Textabschnitt einer Hochschularbeit – und zwar gleichgültig, ob es sich um eine Diplomarbeit,
eine Bachelorthesis oder eine Staatsexamensarbeit handelt. Hier wird der Leser ins Thema eingeführt, hier wird die Fragestellung
formuliert, und hier werden die methodischen Grundsatzentscheidungen getroffen.

Tipp: Man sollte die Einleitung, auch wenn das sogar von Prüferseite geäußert wird, nicht zum Schluss schreiben, sondern sie vorab
verfassen und sie während der Niederschrift als Arbeitsinstrument verwenden. Denn hieran – und nur hieran – zeigt sich, ob das
Untersuchungsvorhaben klappt. Häufig zeigt sich auch, weshalb es nicht so gut klappt.

All das weiß auch der Prüfer. Daher steht im Grunde die Note schon nach der Lektüre der Einleitung fest – zumindest wird es
schwierig sein, den Prüfer davon zu überzeugen, dass es sich doch um eine gute Arbeit handelt, wenn die Einleitung misslungen ist.
Aber wie schreibt man nun eine gute Einleitung? Im Folgenden einige grundlegende Hinweise:

Hilfreich ist eine Aufteilung in vier Teile, die sinnvoll aufeinander aufbauen. Das heißt nicht, dass es nicht auch andere
Möglichkeiten gibt, aber so funktioniert es mit Sicherheit – bei jedem Thema.  Das vorgeschlagene Schema sieht wie folgt aus:

\begin{enumerate}
	\item Hinführung (2 Absätze); diese besteht im Idealfall aus dem
		\begin{enumerate}
			\item Einstieg (etwas, was an die allgemeine Erfahrung anknüpft und unmittelbar ersichtlich ist) und
			\item einem weiteren Absatz, in dem – ausgehend vom Einstieg – auf das eigentliche Thema fokussiert wird.
		\end{enumerate}
		In einer Arbeit über Online-Marketing mit Facebook beispielsweise würde es im 1. Absatz um Online-Marketing allgemein gehen und im 2. Absatz
		auf die besonderen Anforderungen im Zusammenhang mit Facebook verwiesen. (Kontrollfrage: „Worum geht es hier?“)

	\item Fragestellung (1 Satz), auch als Problemstellung oder Forschungsfrage bezeichnet. Dabei handelt es sich im
		Idealfall tatsächlich nur um einen Satz oder einen Fragesatz. Die Fragestellung muss sich organisch aus dem 2. Absatz der Hinführung
		ableiten lassen. Im Idealfall ergibt sich für den Leser selbst angesichts des bisher Vermittelten an exakt dieser Stelle eine Frage, die er
		dann vom Verfasser bzw. der Verfasserin
		formuliert bekommt. Damit einher geht die Vorgabe, dass die Fragestellung wirklich nur exakt eine einzige Frage bzw. ein Forschungsproblem
		betrifft, nicht zwei oder drei. Wenn das passiert, hat man schon an dieser Stelle etwas falsch gemacht.  (Kontrollfrage: „Was ist das
		Problem?“)
	\item Operationalisierung der Fragestellung (3 bis 4 Sätze); dabei wird die Frage beantwortet, welche inhaltlichen und methodischen Aspekte im
		Zusammenhang mit der Klärung der Forschungsfrage wichtig sind. Dabei dürfen dann durchaus weiterführende Fragen gestellt oder Hypothesen
		formuliert werden, die sich aber wiederum
		auf die zentrale Fragestellung zurückführen lassen müssen. (Kontrollfragen: „Welche Überlegungen sind damit verknüpft? Was brauche ich, um
		die Forschungsfrage zu beantworten?“)
	\item Untersuchungsverlauf (pro Kapitel ein – kurzer – Absatz mit Verweis auf die Kapitelnummer); hier wird geklärt, was Inhalt der einzelnen
		Kapitel ist. Neue inhaltliche oder methodische Aspekte sollten hier nicht mehr vorkommen, hier geht es nur noch um das Wie, nicht mehr um
		das Was. (Kontrollfrage: „Wie wird
		vorgegangen?“)
\end{enumerate}

Hinweis: Die Schritte 3 und 4 werden häufig vermischt, oder die Operationalisierung unterbleibt ganz. Wir möchten nochmals deutlich
darauf hinweisen, dass die inhaltlichen, vor allem aber die methodischen und strukturellen Überlegungen in Bezug auf die
Fragestellung ein unverzichtbares Instrument sind, um die grundlegenden Zusammenhänge der Untersuchung zu klären. Demgegenüber
dient der Untersuchungsverlauf lediglich dazu, die Überlegungen im Hinblick auf die Gliederung in eine sinnvolle Abfolge zu
bringen.

Wie gesagt, dieses Schema funktioniert mit jedem Thema – wenn nicht, deutet das eher darauf hin, dass mit dem Thema etwas nicht
stimmt. Konkrete Fragen dazu beantworten wir gern im o:T-Forum. Ansonsten lässt sich auch im Rahmen einer Kurzberatung klären, ob
die Einleitung schlüssig ist. Dazu ist keine Beauftragung eines Lektorats notwendig.

% Don't forget to delete this!!!!

%%%%%%%%%%%%%%%%%%%%%%%%%%%%%%%%%%%%%%%%%%%%%%%%%%%%%%%
%%%%%%%%%%%%%%%%%%%%%%%%%%%%%%%%%%%%%%%%%%%%%%%%%%%%%%%
%%%%%%%%%%%%%%%%%%%%%%%%%%%%%%%%%%%%%%%%%%%%%%%%%%%%%%%
Die\todo{Einleitung mehrfach durchlesen!} Erforschung der Ursachen und der Folgen des Klimawandels, die wachsende
Schwierigkeit bei der Bereitstellung der konventionellen Energien, die Neubewertung der Risiken und technischen Mitteln bei
der Endlagerung von Abfällen der Atomindustrie, die besorgniserregende Erkenntnis der bisherigen Fehlbewertung der
Atomsicherheit \textcolor{red}{werden mit Gewissheit} die politisch beschlossene Förderung der erneuerbaren Energien zu einer
grundlegenden Energieform in den kommenden Jahren in Deutschland und in Europa forcieren.\todo{Reicht so? Guck bitte nach
oben.}

Der Umstieg auf alternative Energien ist mit einigen\todo{Bewertung einfügen.}$\,$ Problemen verbunden. An einigen Orten ist
der Einsatz dieser Technik aus politischer, technischer, ökonomischer oder ökologischer Sicht nicht möglich. Darum weichen
oft die Stromerzeugung und der Strombedarf zeitlich und räumlich voneinander ab. Windkraft im Meer, Wasserkraft in den
Bergen, Sonnenkraft in dem Süden, Geothermie in Island\todo{Der Satz ist scheiße, umschreiben.}, Biomasse Land. Es ist auch
eine Tatsache, dass nur ein Teil der erneuerbaren Energien direkt vom Menschen beeinflusst werden kann. Besonders die Menge
der durch Sonne und Wind gewonnenen Energie schwankt abhängig von der Wetterlage. Die Integration dieser Energie in Netz
führt zur erhöhten Bereitstellung an Regelenergie. Hauptsächlich wird diese Herausforderung durch übermäßige Belastung der
zur Ausregelung geeigneten konventionellen thermischen Kraftwerke gelöst. Durch den regelungsbedingten ineffizienten
Teillastbetrieb und wiederholte An- und Abfahrvorgänge sinkt der Wirkungsgrad und ein höherer Verschleiß der ist die Folge.
Aus langfristiger Sicht wird jedoch\todo[color=red!70,inline]{Das am Ende zur Ende schreiben} eine Investition in
Lastmanagement und in Energiespeicher unentbehrlich sein.

Der Anteil am Bedarf an elektrischen Energie vom Gesamtverbrauch eines Industrielandes für Kälteerzeugung in den Supermärkten
wird in Australien zum Beispiel auf ein Prozent \cite[Seite 8]{australia} und in Schweden auf zwei \cite[Seite
6]{doctor} und \cite{EANRW} Prozent geschätzt\todo{Irgendwie bewerten?}.

Brückentechnologien nicht sicher. Politischer Druck wächst. Es muss in Angriff genommen werden alle Potentiale zur Senkung
des Energeiverbrachs auszuschöpfen.  Rationale Anwendung der Energie in öffentlichen (Kleingewerblichen, Supermärkte)
Einrichtungen oft sehr schwer umsetzbar. Es müssen Reize gefunden werden, die das Verhalten zur rationalen Nutzung der
Energie fördern.

