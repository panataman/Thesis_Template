\chapter{Einleitung}
\label{chap:einleitung}
%%%%%%%%%%%%%%%%%%%%%%%%%%%%%%%%%%%%%%%%%%%%%%%%%%%%%%%
%%%%%%%%%%%%%%%%%%%%%%%%%%%%%%%%%%%%%%%%%%%%%%%%%%%%%%%
%%%%%%%%%%%%%%%%%%%%%%%%%%%%%%%%%%%%%%%%%%%%%%%%%%%%%%%
Die Erforschung der Ursachen und der Folgen des Klimawandels, die wachsende
Schwierigkeit bei der Bereitstellung der konventionellen Energien, die
Neubewertung der Risiken und der technischen Mitteln bei der Endlagerung von
Abfällen der Atomindustrie, die besorgniserregende Erkenntnis der bisherigen
Fehlbewertung der Atomsicherheit treiben die Entwicklung der erneuerbaren
Energien zu einer grundlegenden Energieform in Deutschland noch schneller voran.

Der Umstieg auf alternative Energien ist mit einigen grundlegenden Problemen
verbunden. An einigen Orten ist der Einsatz dieser Technik aus politischer,
technischer, ökonomischer oder ökologischer Sicht nicht möglich. Aus diesem
Grund weichen oft die Stromerzeugung und der Strombedarf zeitlich und räumlich
voneinander ab.  Windkraft im Meer, Wasserkraft in den Bergen, Sonnenkraft in
in den s\"udliche Regionen, Geothermie in seismisch aktiven Gebieten.
Desweiteren kann nur ein Teil der erneuerbaren Energien direkt vom Menschen
beeinflusst werden. Besonders die Menge der durch Sonne und Wind gewonnenen
Energie schwankt abhängig von der Wetterlage. Die Integration dieser Energie in
das Netz führt zur erhöhten Bereitstellung an Regelenergie. Diese
Herausforderung wird zur Zeit haupts\"achlich durch übermäßige Belastung der zur
Ausregelung geeigneten konventionellen thermischen Kraftwerke \"ubernomen. Durch
den regelungsbedingten ineffizienten Teillastbetrieb und wiederholte An- und
Abfahrvorgänge sinkt der Wirkungsgrad und hat einen h\"oheren
Verschlei\ss dieser Kraftwerke zu Folge.

Aus langfristiger Sicht werden Investitionen in Lastmanagement und in
Energiespeicher unverzichtbar. Effiziente Lastmanagemant und
Energiespeicherkonzepte k\"onnen helfen diesen Problemen langfristig
entgegenzuwirken.

Die Intention der Arbeit ist es, im Ramen der Bereitstellung der Regelleistung
f\"ur erneuerbaren Energien durch den Einsatz der Verfahren und Techniken des
objektorientierten Entwurfs ein \matlab-Programm zur Simulation des variablen
Lastverhaltens von Kältelasten mit Kältespeichern im Energieversorgungsnetz zu
entwerfen. An erster Stelle im \cref{chap:theorie} wird der Einsatz der
Suparm\"arkte als K\"altespeischer diskutiert. Es folgt eine kurze Einf\"uhrung
in die objektorientierte Programmierung mit \matlab. Anschlie\ss end werden die
physikalischen Grundlagen der K\"altetechnik erkl\"art. Im \cref{chap:mode} wird
der L\"osungsweg vom objektorientierten Entwurf bis zum Programm-Code erkl\"art.

Desweiteren wird mit Hilfe der K\"altelast spezifizierender
Konfigurationsdateien\footnote{ Der Entwurf der Konfigurationsdateien basiert
auf der Spezifikation der Parameter durch Caroline M\"oller. Der Titel ihrer
Arbeit, die ebenfalls im Fachgebiet Energieversorgungsnetze und Integration
erneuerbarer Energien an der TU Berlin entstand und die Spezifikation einer
K\"altelast zum Inhalt hat, heißt: Spezifikation und Simulation einer Kältelast
mit Kältespeicher im Energieversorgungsnetz.} das variable Verhalten der bei
Variation der Anzahl der K\"altespeicher im Enerergieversorgungsnetz mit dem
Programm simuliert und der Einfluss auf das Energieversorgungsnetz untersucht.
Es werden ausschlie\ss lich K\"alteanlagen modelliert, die in
Superm\"arkten zum Einsatz kommen. Das Ergebnis der Simulation, der
Verbrauch einer variabel gef\"uhrten Supermarktkette auf der Basis der einzelnen
in den Superm\"arkten eingesetzten K\"alteanlagen, wird im \cref{chp:sue}
vorgestellt.

