\chapter{Einleitung}
\label{chap:einleitung}
%%%%%%%%%%%%%%%%%%%%%%%%%%%%%%%%%%%%%%%%%%%%%%%%%%%%%%%
%%%%%%%%%%%%%%%%%%%%%%%%%%%%%%%%%%%%%%%%%%%%%%%%%%%%%%%
%%%%%%%%%%%%%%%%%%%%%%%%%%%%%%%%%%%%%%%%%%%%%%%%%%%%%%%
Die\todo{Einleitung mehrfach durchlesen!} Erforschung der Ursachen und der
Folgen des Klimawandels, die wachsende Schwierigkeit bei der Bereitstellung der
konventionellen Energien, die Neubewertung der Risiken und technischen Mitteln
bei der Endlagerung von Abfällen der Atomindustrie, die besorgniserregende
Erkenntnis der bisherigen Fehlbewertung der Atomsicherheit
\textcolor{red}{werden gewiss} die politisch beschlossene Förderung der
erneuerbaren Energien zu einer grundlegenden Energieform in den kommenden Jahren
in Deutschland und in Europa forcieren.\todo{Reicht so? Guck bitte nach oben.}

Der Umstieg auf alternative Energien ist mit einigen\todo{Bewertung
einfügen.}$\,$ Problemen verbunden. An einigen Orten ist der Einsatz dieser
Technik aus politischer, technischer, ökonomischer oder ökologischer Sicht nicht
möglich. Darum weichen oft die Stromerzeugung und der Strombedarf zeitlich und
räumlich voneinander ab. Windkraft im Meer, Wasserkraft in den Bergen,
Sonnenkraft in dem Süden, Geothermie in Island. Desweiteren kann nur ein Teil
der erneuerbaren Energien direkt vom Menschen beeinflusst werden. Besonders die
Menge der durch Sonne und Wind gewonnenen Energie schwankt abhängig von der
Wetterlage. Die Integration dieser Energie in das Netz führt zur erhöhten
Bereitstellung an Regelenergie. Diese Herausforderung Hauptsächlich wird durch
übermäßige Belastung der zur Ausregelung geeigneten konventionellen thermischen
Kraftwerke gelöst. Durch den regelungsbedingten ineffizienten Teillastbetrieb
und wiederholte An- und Abfahrvorgänge sinkt der Wirkungsgrad und ein höherer
Verschleiß der ist die Folge. Aus langfristiger Sicht wird
jedoch\todo[color=red!70,inline]{Das am Ende zur Ende schreiben} eine
Investition in Lastmanagement und in Energiespeicher unentbehrlich sein.
Effiziente Lastmanagemant und Energiespeicherkonzepte k\"onnen helfen diesen
Problemen langfristig entgegenzuwirken. Brückentechnologien nicht sicher.
Politischer Druck wächst.
