\chapter{Ausblicke}

Die Auswertung in Abschnitt 5.1 zeigt die Auswirkungen einer Kooperation zwi-
schen einem Windpark und den Kälteanlagen eines Supermarktes. Die Kälte-
speicher des Supermarkts nehmen überschüssige Energie auf, senken ihre Tem-
peratur und überbrücken Zeiträume mit wenig zur Verfügung stehender Energie,
indem sie sich wieder erwärmen. Auch die Notfallversorgung mit Energie aus
dem Netz funktioniert und ein Überschreiten der oberen Temperaturgrenze wird
verhindert. In Abschnitt 5.2 ist gezeigt worden, dass durch das direkte Zusam-
menwirken von Windpark und Supermarkt Regelenergie eingespart wird. Der
folgende Abschnitt beinhaltet eine Bewertung der Ergebnisse dieser Arbeit und
gibt gleichzeitig einen Ausblick für künftige Arbeiten an derselben Problemstel-
lung.

Die eingesparte Regelenergie ist 1 \% der Gesamtregelenergie. Die Kältespei-
cher haben damit nur einen geringen Einfluß. Gezeigt wurde jedoch, dass die
Möglichkeit besteht, Regelenergie durch den Einsatz von Kälteanlagen als flexi-
ble Verbraucher einzusparen. Bei der Bewertung ist überdies zu beachten, dass
mit einem Modell und einer groben Hochrechnung für die Vattenfall-Regelzone
gerechnet wurde. Dieses Modell stellt nicht die Realität dar, was bei einer quanti-
tativen Aussage zur berechneten Regelenergieeinsparung berücksichtigt werden
muss.

Im folgenden werden Verbesserungen und Möglichkeiten zur Weiterentwicklung
vorgeschlagen, um die Untersuchung auszubauen und realistischer zu gestal-
ten. Die in dieser Arbeit vorgestellte Berechnungsmethode kann eine Grundlage
für die Untersuchung weiterer Supermärkte sein. Verbesserungen sind möglich
durch das Einbeziehen weiterer Fakten und Parameter in das Supermarktmodell.
So können z. B. Spitzenlastzeiten berücksichtigt und damit die Verlustleistun-
gen nach Tageszeiten differenzierter betrachtet werden. Auch die Verluste in
der Nacht sollten mit zusätzlichen, realistischen Werten ergänzt werden. Inter-
essant wäre auch eine Untersuchung unter der Berücksichtigung, dass die An-
zahl der Lebensmittel in den Kühlstellen nicht konstant bleibt und sich dadurch
die Speicherkapazität ständig ändert. Bei der in dieser Arbeit durchgeführten
Untersuchung wird die Nennleistung der Kälteanlagen außer Acht gelassen. Es
wird davon ausgegangen, dass jegliche Leistung, sobald es die Kapazität der
Kältespeicher zulässt, abgenommen werden kann. Eine Verbesserung, die die
installierte Leistung der Kälteanlagen als Kriterium für die Aufnahmefähigkeit
berücksichtigt, ist daher notwendig. Zudem sollten in einer weiteren detaillierte-
ren Hochrechnung verschiedene Supermarktvarianten betrachtet und die Anzahl
der Supermärkte in der Simulation, der Anzahl der Supermärkte in der gesamten
Vattenfall-Regelzone angepasst werden.

Diese Arbeit wurde ausschließlich mit theoretischen Berechnungen erstellt. Mes-
sungen können daher sinnvoll sein, um das Verhalten einzelner Parameter genau-
er beurteilen zu können. Mit Messungen könnte der Energieverbrauch zu jeder
Tageszeit ermittelt werden und damit eine gründlichere Betrachtung der Ver-
lustleistungen erfolgen. Auch die Leistungszahlen und die Betriebsstunden der
Kälteanlagen können genau bestimmt werden. Der Zusammenhang zwischen zu-
und abgeführter Wärmeenergie und dem Temperaturverlauf sollte ebenfalls mit
Messungen analysiert werden. Untersuchungen zum Temperaturverlauf in den
Kühlstellen und zur thermischen Trägheit der Lebensmittel würden außerdem
Aufschluss darüber geben, welche Rolle die Lebensmittel tatsächlich bei der Be-
wertung als Kältespeicher spielen. Ein exaktes Ermitteln der Lebensmittel oder
ein Abstrahieren, wie es in dieser Arbeit gehandhabt wurden, wären die Alterna-
tiven, die sich aus den Ergebnissen dieser Messungen ergeben.
Eine Weiterentwicklung der Untersuchung von Supermarktkälteanlagen als Käl-
tespeicher wäre eine interessante Aufgabenstellung. Die Möglichkeit der Nut-
zung von externen Eisspeichern sollte in Betracht gezogen werden, um hohe
Temperaturschwankungen in den Kühlstellen des Supermarktes zu vermeiden
und die Aufnahmekapazität zu erhöhen. Für die Idee der Kooperation zwischen
Windpark und Supermarkt sollten optimale Parameter gefunden werden. Zudem
können andere Möglichkeiten der Kooperation, z. B. mit einer Planung der zu
speichernden Energie durch die Orientierung an den aktuellen Energiepreisen,
Anwendung finden.

