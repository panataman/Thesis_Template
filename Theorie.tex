\chapter{Theorie}
\label{chap:theorie}
\minitoc

\section{Erneuerbare Energien im Energieversorgungsnetz}

Die Erneuerbaren Energien \"ubernehmen in Deutschland mit jedem Jahr mehr Anteil
an der Elektrizit\"atsversorgung. Mehr als die H\"alfte der Erneuerbaren Energie
wird durch Wind und Photovoltaik erzeugt. Diese Energiequellen sind jedoch im
hohen Grad witterungsabh\"angig und k\"onnen stark fluktuieren. Dadurch kann die
Menge der zur Verf\"ugung stehenden Energie von der Nachfrage zeitlich enorm
abweichen. Der sichere Betrieb des Stromnetzes bei einer konstanten Frequenz von
$50\,Hz$ kann dadurch gef\"ardet werden. Zus\"atzliche Bereitstellung an
Regelleistung\todo{Regelleistung vs. Regelenergie, was ist richtig?} wird dadurch
notwendig. Au\ss erdem erfolgt die Erstellung der Fahrlpl\"ane f\"ur den Einsatz
der konventionellen Kraftwerke zwangsweise auf den fehlerbehafteten
Vortagsprognosen \"uber die Menge an Leistung aus Erneuerbaren Energien.
Die G\"ute der Prognose bestimmt direkt den Regel- und den Reservebedarf.

\subsection*{Wetter- und Windprognose}

Um eine exakte Prognose über die WEA-Leistung zu erhalten\todo{Das ist
geklaut bie \cite{prognose_doctor}!  Beachte!}, ist daher eine möglichst genaue
Vorhersage der Windverhältnisse an der Anlage bzw. dem Windpark zu treffen.
Derzeit gibt es Abweichungen von 3 \% (2-Stunden-Prognose·) bzw. 7 \%
(24-Stunden- Prognose) zwischen der prognostizierten und der eingespeisten
Leistung ([ISET], [DENA]). Diese Werte werden sich in Zukunft verbessern, jedoch
wird es immer eine Abweichung zwischen der Prognose und dem realen Verlauf
geben. Auch bei einer augenscheinlich sehr guten Korrelation zwischen der
Prognoseleistung und der real erzeugten Leistung einer Anlage erhebliche
Abweichungen geben kann. In dem gewählten Beispiel treten Leistungsunterschiede
von bis zu 40 \% der Nennleistung der Anlage auf.  Dies kann vorkommen, wenn etwa
die Gradienten der Leistung hohe Werte erreichen (z.B. bei Windböen) und
zeitgleich die Prognose um einige Minuten abweicht.  Die
Energieversorgungsunternehmen sind bestrebt, eine planbare Betriebsführung ihrer
Kraftwerke und Netze zu haben. Dies ist sowohl technisch als auch ökonomisch von
Vorteil. Dazu ist es notwendig, dass die Kraftwerke ein prognostizierbares
Einspeisever- halten haben bzw. am besten einem bestimmten Fahrplan nachfahren.
Windenergieanlagen können dies durch den wechselhaften Charakter des Windes
nicht ohne weiteres leisten.  Die Fluktuationen einer WEA sind sowohl
jahreszeitlich als auch innerhalb des Tages verschieden. Dazu können bei
schlechten Prognosen hohe Leistungsänderungen bezogen auf die vorausgesagte
Einspeisung auftreten. Eine Möglichkeit, dies auszugleichen, wäre die Regelung
der Anlagen selbst. Diese würden gedrosselt betrieben werden und bei Bedarf
selbst die Abweichungen von ihrer prognostizierten Einspeisung kompensieren.
Dadurch geht jedoch wertvolle elektrische Energie verloren, weil der gedrosselte
Anteil nicht in das Netz eingespeist wird. So bleibt nur der Ausgleich durch
Kraftwerke bzw. dem Energiemarkt. Abhilfe können jedoch auch Energiespeicher
bieten, die die Fluktuationen durch Ein- und Ausspeichern von Energie
ausgleichen.

%%%%%%%%%%%%%%%%%%%%%%%%%%%%%%%%%%%%%%%%%%%%%%%%%%%%%%%%%%%%%%%%%%%%%%%%%%%%%%%%
%%%%%%%%%%%%%%%%%%%%%%%%%%%%%%%%%%%%%%%%%%%%%%%%%%%%%%%%%%%%%%%%%%%%%%%%%%%%%%%%

\subsubsection*{Veränderliche Lastflüsse}

Aufgrund des unstetigen Windes kann nie mit völliger Sicherheit
vorausgesagt werden, wie sich die Einspeisung von WEA zu einem bestimmten
Zeitpunkt verhält. Bei hohen Windstärken besteht zudem die Gefahr, dass Anlagen
abgeschaltet werden müssen. Sollte dies bei einem Sturm in der Nordsee der Fall
sein, könnten innerhalb von wenigen Minuten alle offshore installierten Anlagen
vom Netz gehen, womit deren volle Leistung fehlen würde.

%%%%%%%%%%%%%%%%%%%%%%%%%%%%%%%%%%%%%%%%%%%%%%%%%%%%%%%%%%%%%%%%%%%%%%%%%%%%%%%%
%%%%%%%%%%%%%%%%%%%%%%%%%%%%%%%%%%%%%%%%%%%%%%%%%%%%%%%%%%%%%%%%%%%%%%%%%%%%%%%%

\subsubsection*{Erhöhter Regelleistungsbedarf}

Zum Ausgleich unvorhersehbarer Differenzen zwischen Prognose und realer
Windenergie- einspeisung benötigt man Regelleistung. Die Höhe des RL-Bedarfs ist
direkt abhängig von der Güte der Windprognose. Je besser diese ist, desto
kleiner sind die Abweichungen und desto weniger RL muss bereitgehalten
werden.\cite{prognose_doctor}

Wachsender Anteil der Windenergie wird zunehmenden Bedarf an Regelleistung
verursachen.  Implikation für die konventionelle Strom erzeugung bei konstantem
Niveau der Versorgungszuverlässigkeit:

\begin{itemize}
	\item Konventionelle Kraftwerke müssen zunehmend im Teillastmodus
	betrieben werden;
	\item Pumpspeicherkraftwerke sind verstärkt als Regelkraftwerke
	einzusetzen;
\end{itemize}

Zunehmender Anteil der Windenergie an der Stromerzeugung ist nicht mit
entsprechenden Kapazitätseffekten verbunden.

%%%%%%%%%%%%%%%%%%%%%%%%%%%%%%%%%%%%%%%%%%%%%%%%%%%%%%%%%%%%%%%%%%%%%%%%%%%%%%%%
%%%%%%%%%%%%%%%%%%%%%%%%%%%%%%%%%%%%%%%%%%%%%%%%%%%%%%%%%%%%%%%%%%%%%%%%%%%%%%%%

\section{K\"alteanlagen im Supermarkt: Potentiale und Besonderheiten\todo{Es ist
noch nicht ausformuliert! Das ist noch sehr Plagiatlastig!}}

Bei einem durchschnittlichen Supermarkt entfallen 62 \% des Stromverbrauchs auf
Kühlen und Tiefkühlen. Ein großer Supermarkt verbraucht zehnmal so viel Strom
wie ein Bürogebäude gleicher Größe (Quelle: Zahlen und Tabellen, Novem).
Fast 2/3 des Stromverbrauchs in einem Supermarkt entfallen auf das Kühlen und
Tiefkühlen von Produkten. Bei Supermärkten sind die Verbrauchszahlen hoch, die
meisten Verbräuche liegen zwischen 300 und 500 MWh auf Jahresbasis.

%%%%%%%%%%%%%%%%%%%%%%%%%%%%%%%%%%%%%%%%%%%%%%%%%%%%%%%%%%%%%%%%%%%%%%%%%%%%%%%%
%%%%%%%%%%%%%%%%%%%%%%%%%%%%%%%%%%%%%%%%%%%%%%%%%%%%%%%%%%%%%%%%%%%%%%%%%%%%%%%%

\section{Objektorientierte Programmierung mit Matlab}

Seit Ende des letzten Jahrhunderts herrscht in der Fachliteratur für Informatik
die Meinung, dass der Einsatz von objektorientierten Techniken Programme
hervorbringt, die im Vergleich \textit{einfacher erweiterbar, besser testbar}
und \textit{besser wartbar} sind\todo{SIND, das ist hier nicht gut.}. Dabei wird
ein Verfahren angewendet, nach dem große Systeme in kleinere Teile des Ganzen
zerlegt werden. Programme lassen sich dadurch im Allgemeinen mit weniger Aufwand
und kleineren Fehlerwahrscheinlichkeit programmieren. Inspiriert durch die
Vorgänge aus der realen Welt, werden die Abläufe durch operierende Objekte
vorgestellt, die Aufträge erledigen und vergeben können. Die wesentlichen
Eigenschaften der objektorientierten Programmierung, kurz
OOP\abvz{OOP}{objektorientierte Programmierung}, sind die Datenkapselung, die
Polymorphie und die Vererbung.\footnote{ Ausführliche Informationen dazu findet
man z.B.  in \cite{OOP},\cite{java} oder \cite{python}.}

Wichtige im oberen Teil des Kapitels verwendete Definitionen werden nun n\"aher
vorgestellt\todo{Einleitungssatz noch nicht gut, oder?}. Anhand eines
Beispielcodes erfolgt anschlie\ss end eine kurze Pr\"asentation einiger
wichtiger Schl\"usselw\"orter der Syntax, die bei der OOP mit \matlab verwendet
werden.


%%%%%%%%%%%%%%%%%%%%%%%%%%%%%%%%%%%%%%%%%%%%%%%%%%%%%%%%%%%%%%%%%%%%%%%%%%%%%%%%
%%%%%%%%%%%%%%%%%%%%%%%%%%%%%%%%%%%%%%%%%%%%%%%%%%%%%%%%%%%%%%%%%%%%%%%%%%%%%%%%

\begin{description}

	\item[\textbf{Klasse}] Eine Klasse ist ein Instrument der
	Programmierung zur Erfassung von charakteristischen Eigenschaften
	zusammenh\"angende!!r Objekte. Die Definition der Strukturen der Objekte
	erfolgt durch Klassen.

	\item[\textbf{Objekt}] Ein Objekt ist ein konkretes Exemplar einer
	Klasse.

	\item[Datenkapselung] Der Zugriff auf Attribute einer Klasse erfolgt
	gew\"ohnlich durch ihre Methoden, die Kommunikationsschnittstellen
	darstellen. Dieses Verfahren bezeichnet man Datenkapselung.

	\item[Polymorphie] Wenn einem Objekt einer bestimmten Klasse die
	Objektvariablen einer anderen Klasse zugewiesen werden k\"onnen, spricht
	man von Polymorphie.

	\item[Vererbung] Die Vererbung erm\"oglicht durch Ver\"nderung der
	bestehenden Klassen neue Klassen zu erstellen. Die grundlegenden
	Programmteile der bestehenden Klasse werden zwangsl\"aufig
	\"ubernomen.\footnote{ Ausf\"urliche Informationen dazu findet man z.B.
	in \cite{pepperOOP}.}

\end{description}


\begin{lstlisting}[float=h!,frame=none]
	classdef (Attributes) class_name < super_class % class definition
		properties (Attributes) % first property block
			PropertyName
		end % end of properties block
		% additionally here can be another methods block with 
		% specifying by another Attributes
		methods (Attributes) 
			function obj = class_name(obj,a) % consturctor
				obj.PropertyName = a;
			end
			function [X] = second_function(obj)
				X = obj.PropertyName + 1;
			end
		end
		% additionally here can be another methods block with 
		% specifying by another Attributes
	end % end of classdef block
\end{lstlisting}

%%%%%%%%%%%%%%%%%%%%%%%%%%%%%%%%%%%%%%%%%%%%%%%%%%%%%%%%%%%%%%%%%%%%%%%%%%%%%%%%
%%%%%%%%%%%%%%%%%%%%%%%%%%%%%%%%%%%%%%%%%%%%%%%%%%%%%%%%%%%%%%%%%%%%%%%%%%%%%%%%

\section{Überblick über die mathematische Zusammenhänge}

In diesem Kapitel wird ein knapper \"Uberblick über die mathematischen und
physikalischen Zusammenhänge, die bei der Entwicklung eines Modellsupermarktes
zwingend beachtet werden müssen, vorgestellt\todo{Der Satz ist total scheiße
man}. Es beinhaltet eine Zusammenfassung des mathematischen Konstruktes aus dem
Kapitel 3 der Diplomarbeit von Caroline Möller \cite{caro}, der bei der
Entwicklung des Programms zu Grunde gelegt wurde.

Die primäre Aufgabe der Kühleinheiten in einem Supermarkt besteht in der Regel
darin, Lebensmitteltemperatur unter die Zimmertemperatur zu bringen und diese
dabei zu halten\todo{Hört sich komisch an!}. Körper mit unterschiedlicher
Temperatur sind bestrebt, wenn sie thermisch von einander nicht vollkommen
isoliert sind, durch gegenseitige Wechselwirkung ihre Temperaturen anzugleichen,
sodass ein Wärmegleichgewicht entsteht, wobei der natürliche, selbständige
Wärmefluss immer von einem Körper mit höheren Temperatur in Richtung des Körpers
mit kleineren Temperatur stattfindet. Um eine negative Temperaturänderung
herzustellen und diese auch zu halten, muss die eindringende Wärmeenergie
ständig in der selben Höhe abgeführt werden, damit die Temperatur konstant
bleibt. Diese Energiemenge pro Zeiteinheit wird als Kälteleistung bezeichnet.
Eine Abweichung von dieser Menge führt zum Steigen der Temperatur, wenn weniger
und zum sinken der Temperatur wenn mehr abgeführt wird. Um diesen Kühlkreislauf
aufrecht zu erhalten, muss Leistung aufgewendet werden\footnote{ Eine
detailierte Beschreibung dieser Prozesse in einer Kopressionskälteanlage und
Spezifikation ist nicht Gegenstand dieser Arbeit.  Ausführliche Informationen
dazu findet man z.B.  in \cite{caro, doctor, TAB_A1}.}.

Das wird ausformuliert und mit einem Bild visualisiert\todo{bald schnell
machen}.
\begin{itemize}
	\item Temperaturunterschied in Kelvin $K$
	\item Wärmeausgleich (Verlust an Kälte)
	\item Kälteenergie gespeichert in Körpern mit einer bestimmten
	spezifischen Wärmekapazität
	\item Kälteenergie gespeichert in Masse
\end{itemize}
\begin{figure}\caption{ Modellgrundlage}
	\missingfigure{Modellgrundlage}
\end{figure}

Aufnahme der Wärmeenergie und der spezifischen Wärmekapazität dieser
Substanzmasse ergibt die Temperaturdifferenz $\Delta\:t$.

\begin{equation}
	\Delta\:t = \frac{Q}{m\cdot c}
\label{tdif}
\end{equation}

\begin{description}[\dth]

	\item[$\Delta\:t$] Temperaturdifferenz in Kelvin $K$
	\item[$Q$] Eindringende Wärmeenergie in $kJ$
	\item[$m$] Substanzmasse zur Aufnahme der Wärmeenergie in $kg$
	\item[$c$] Spezifische Wärmekapazität der Substanzmasse in $\frac{kJ}{kg
		\cdot K}$

\end{description}
\vspace{0.5cm}

Die installierte Kälteleistung multipliziert mit der täglichen Betriebszeit muss
zwangsweise größer oder gleich dem stündlichen Kältebedarf multipliziert mit der
Tagesstundenzahl sein.

\begin{equation}
	\pinstall = \frac{24\,h}{ \tau_{B} }  \cdot \pkalt \label{pinstall}
\end{equation}

\begin{description}[\dth]

	\item[$\pinstall$] Installierte Kälteleistung in $kW$
	\item[$\tau_{B}$] Tägliche Betriebszeit in $h$
	\item[$\pkalt$] Kältebedarf in $kW$

\end{description}
\vspace{0.5cm}

Die Transmissionswärmeleistung wird aus der Multiplikation der Fläche
der wärmeübertragenden Wänd mit ihrem spezifischen Wärmedurchgangskoeffizient
und der Temperaturdifferenz zwischen der Kühlraumtemperatur und der
Umgebungstemperatur berechnet.

\begin{equation}
	\ptrans = A \cdot k \cdot \Delta t
	\label{ptrans}
\end{equation}

\begin{description}[\dth]

	\item[$\ptrans$] Transmissionswärmeleistung in $kW$
	\item[$A$] Fläche in $m^2$
	\item[$k$] Wärmedurchgangskoeffiziente in $\frac{W}{m^2 \cdot K}$
	\item[$\Delta\: t$] Temperaturdifferenz in $K$

\end{description}
\vspace{0.5cm}

Der Zusammenhang zwischen der aufgewendeten elektrischen Antriebsleistung $P$
eines Verdichters in einer Kompressionskälteanlage und der genutzten
Kälteleistung ${\dot{Q}}_0$ wird durch die Kältezahl $\epsilon$ wiedergegeben.
Die Leistungszahl wird in die zweite Spalte im Array (vergl. Zeile 4 im
\matref{fridge}) eingetragen.

\begin{equation}
	\epsilon = \frac{\pkalt}{P}
\label{epsilon}
\end{equation}

\begin{description}[\dth]

	\item[$\epsilon$] Leistungszahl (einheitenlos)
	\item[$\pkalt$] Kälteleistungsbedarf in $kW$
	\item[$P$] elektrische Verdichterantriebsleistung in einer
		Kopressionskälteanlage in $kW$

\end{description}
\vspace{0.5cm}

Die Öffnungszeit des Supermarkts hat einen spürbaren Einfluss auf die Größe der
Wärmeverluste. In der Literatur wird der Nachtverbrauch mit $10\%$ bis $20\%$
des Tagesverbrauchs angegeben\cite{kauffeld}.  In der Nacht fallen keine
zusätzlichen Verluste zum Beispiel durch Licht, Körperwärme oder
Türöffnungszeiten, sodass nur Transmissionsverluste bei der Berechnung beachtet
werden.

\begin{equation}
	\dot{Q}_{Nacht}=\ptrans
\label{pnacht}
\end{equation}

\begin{description}[\dth]

	\item[$\pnacht$] Leistungsbedarf in der Nacht in $kW$
	\item[$\ptrans$] Transmissionswärmeleistung in $kW$

\end{description}
\vspace{0.5cm}


Der Leistungsbedarf am Tag ergibt sich aus der Summe des Tagesmehrbedarfs und
der Transmissionswärmeleistung. Aus Gründen der Vereinfachung wird
Tagesmehrbedarf als weitgehend konstant angenommen.

\begin{equation}
	\ptag = \pmehr + \ptrans
\label{ptag}
\end{equation}

\begin{description}[\dth]

	\item[$\ptag$] Leistungsbedarf am Tag in $kW$
	\item[$\pmehr$] Mehrbedarf am Tag in $kW$
	\item[$\ptrans$] Transmissionswärmeleistung in $kW$

\end{description}
\vspace{0.5cm}

Die Berechnung der mittleren Transmissionswärmeleistung erfolgt durch die
Multiplikation der Differenz zwischen der Umgebungstemperatur und der mittleren
Kühlraumtemperatur mit dem Wärmedurchgangskoeffizienten und der Fläche der
wärmeübertragenden Wände.

\begin{equation}
	\aptrans = A \cdot k \cdot \left( t_{amb} -
	\overline{t}_{KR} \right) \label{aptrans}
\end{equation}

\begin{description}[\dth]

	\item[$\aptrans$] mittlere Transmissionswärmeleistung in $kW$
	\item[$A$] Fläche in $m^2$
	\item[$k$] Wärmedurchgangskoeffiziente in $\frac{W}{m^2 \cdot K}$
	\item[$t_{amb}$] Umbgebungstemperatur in $^{\circ}C$
	\item[$\overline{t}_{KR}$] mittlere Kühlraumtemperatur in
		$^{\circ}C$
\end{description}
\vspace{0.5cm}

Ist f\"ur eine K\"alteanlage der K\"alteleistungsbedarf bekannt so wird der
Tagesmehrbedarf an K\"alteleistung ermittelt, indem vom Produkt des
K\"alteleistungsbarfes f\"ur $24\;h$ mit dem Faktor f\"ur
K\"altebedarfsabsenkung die mittlere Transmissionswärmeleistung abgezogen wird.

\begin{equation}
	\pmehr = \pkalt \cdot K - \aptrans
\label{pmehr}
\end{equation}

\begin{description}[\dth]

	\item[$\pmehr$] Mehrbedarf am Tag in $kW$
	\item[$\pkalt$] Kälteleistungsbedarf in $kW$
	\item[$K$] Faktor für Kältebedarfsabsenkung (einheitenlos)
	\item[$\aptrans$] mittlere Transmissionswärmeleistung in $kW$

\end{description}
\vspace{0.5cm}

Ist der Kälteleistungsbedarf nicht bekannt, wie zum Beispiel bei steckerfertigen
Geräten, kann der Mehrbedarf am Tag \"uber den Wert Verdichterarbeit pro 24
Stunden $\lverd$ für den gesamten Kälteverbraucher ermittelt werden.

Das Produkt aus dem spezifischen Energieverbrauch mit dem Faktor für
Kältebedarfsabsenkung, dem Verdichteranteil und der Anzahl der Geräte ergibt die
Verdichterarbeit.

\begin{equation}
	\lverd = \lspez \cdot K \cdot v \cdot n
\label{lverd}
\end{equation}

\begin{description}[\dth]

	\item[$\lverd$] Verdichterarbeit pro $24\,h$ in $\frac{kWh}{24h}$
	\item[$\lspez$] spezifischer Energieverbrauch pro $24\,h$ in
		$\frac{kWh}{24h}$
	\item[$K$] Faktor für Kältebedarfsabsenkung (einheitenlos)
	\item[$v$] Verdichteranteil (einheitenlos)
	\item[$n$] Anzahl der Geräte

\end{description}
\vspace{0.5cm}

Im Folgenden muss die Verdichterarbeit zum Abf\"uhren von zus\"atzlich
eindringenden W\"aremeenergie in der \"Offnungszeit berechnet werden, die einem
weiteren Schritt in W\"armeleistung $\pmehr$ umgewandelt wird.

\begin{equation}
	\lmehr = \lverd - \frac{\ptrans}{\epsilon} \cdot 24h
\label{lmehr}
\end{equation}

\begin{description}[\dth]

	\item[$\lmehr$] Verdichterarbeit zum Abführen von eindringenden
		Wärmeenergie $\emehr$ in $kWh$ in der \"Offnungszeit
	\item[$\lverd$] Verdichterarbeit pro $24\,h$ in $\frac{kWh}{24h}$
	\item[$\ptrans$] Transmissionswärmeleistung in $kW$
	\item[$\epsilon$] Leistungszahl (einheitenlos)

\end{description}
\vspace{0.5cm}

Die Umrechnung von $\lmehr$ in $\pmehr$ erfolgt mit Hilfe der Leistungszahl
$\epsilon$. Die Verdichterarbeit $\lmehr$ wird mit $\epsilon$ multipliziert.
Es wird angenommen, dass der Mehrbedarf in den rund zw\"olf Stunden der
\"Offnungszeit einf\"a

\begin{equation}
	\pmehr = \frac{\lmehr}{12h} \cdot \epsilon
\label{<++>}
\end{equation}

\begin{description}[\dth]

	\item[$\pmehr$] Mehrbedarf am Tag in $kW$
	\item[$\lmehr$] Verdichterarbeit zum Abführen von eindringenden
		Wärmeenergie $\emehr$ in $kWh$ in der \"Offnungszeit
	\item[$\epsilon$] Leistungszahl (einheitenlos)

\end{description}
\vspace{0.5cm}
berechnen.

Dabei entspricht $\pnacht$ der mittleren Transmissionswärmeleistung $\ptrans$.
$\ptag$ ist der Kältebedarf $\pkalt$, multipliziert mit dem Faktor für die
Kältebedarfsabsenkung $K$ bei Geräten, bei denen der Kältebedarf gegeben ist.
Bei steckerfertigen Geräten ist $\ptag$ die Summe aus dem Mehrbedarf an Leistung
am Tag $\pmehr$ und der mittleren Transmissionswärmeleistung $\ptrans$.

Mit den Wärmeverlusten, die am Tag und in der Nacht in unterschiedlicher Größe
auftreten, wird für jeden Zeitschritt die Zeit bis zum kritischen
Temperaturmaximum bestimmt. Diese Zeit braucht das Programm, um den Einsatz der
Supermarktkälteanlagen als Speicher zu planen. Mit der Gleichung

\begin{equation}
	\tau_{krit}(i) = \frac{m \cdot c \cdot (t_{max} -
		t(i))}{\aptranslog}
\label{taukn}
\end{equation}

\begin{description}[\dth]

	\item[$\tau_{krit}$] Zeit bis zur kritischen Temperatur in $h$
	\item[$m$] Substanzmasse zur Aufnahme der Wärmeenergie in $kg$
	\item[$c$] Spezifische Wärmekapazität der Substanzmasse in $\frac{kJ}{kg
		\cdot K}$
	\item[$t_{max}$] obere Temperaturgrenze in $ ^{\circ} C $
	\item[$t(i)$] Temperatur zur Stunde $i$ in $ ^{\circ} C $
	\item[$\aptranslog$] logrithmierte Mittelwert der
		Transmissionswärmeleistung in $kW$|
\end{description}
\vspace{0.5cm}


wird die Zeit $\tau_{krit}$ für jeden Zeitpunkt $i$ berechnet, wobei
$\overline{\dot{Q}}_{Tr_{ln}}$ der logarithmische Mittelwert ist, der sich aus
den Transmissionswärmeleistungen zum jeweils aktuellen Zeitpunkt $i$ mit der
Temperatur $t(i)$ und den Transmissionswärmeleistungen zum Zeitpunkt, an dem der
Kühlinnenraum die maximale Temperatur $t_{max}$ erreicht hätte, berechnet. Am
Tag müssen die restlichen Verluste $\pmehr$ zusätzlich zu den
Transmissionwärmeverlusten für die Berechnung der Zeit bis zur kritischen
Temperatur berücksichtigt werden, wodurch sich folgende Gleichung ergibt:

\begin{equation}
	\tau_{krit}(i) = \frac{m \cdot c \cdot (t_{max} -
		t(i))}{{\overline{\dot{Q}}_{Tr_{ln}}} + \pmehr}
\label{taukd}
\end{equation}

\begin{description}[\dth]

	\item[$\tau_{krit}$] Zeit bis zur kritischen Temperatur in $h$
	\item[$m$] Substanzmasse zur Aufnahme der Wärmeenergie in $kg$
	\item[$c$] Spezifische Wärmekapazität der Substanzmasse in $\frac{kJ}{kg
		\cdot K}$
	\item[$t_{max}$] obere Temperaturgrenze in $ ^{\circ} C $
	\item[$t(i)$] Temperatur zur Stunde $i$ in $ ^{\circ} C $
	\item[$\aptranslog$] logrithmierte Mittelwert der
		Transmissionswärmeleistung in $kW$
	\item[$\pmehr$] Mehrbedarf am Tag in $kW$

\end{description}
\vspace{0.5cm}

Die Zeit $\tau_{krit}$ ist abhängig vom Anstieg der Temperaturen und dieser
wiederum von den eindringenden Wärmelasten.

Um den Temperaturausgleich in den Lebensmitteln im Algorithmus zu
berücksichtigen, werden deshalb die zu- und abgeführten Wärmeenergiemengen bei
der Berechnung der Temperatur für jeden Zeitschritt stets mit dem Faktor 0,8
multipliziert.  Die Gleichung zur stündlichen Berechnung der aktuellen
Temperatur ist damit:

\begin{equation}
	t(i+1) = 0.8 \cdot \frac{Q_v - Q_{ab}}{m \cdot c} + t(i)
\label{tns}
\end{equation}

\begin{description}[\dth]

	\item[$t$] Temperatur in $^{\circ} C$
	\item[$Q_v$] eindringende Verlustwärmemenge in $kJ$
	\item[$Q_{ab}$] abführende Wärmemenge in $kJ$

\end{description}
\vspace{0.5cm}

wobei $Q_v$ die aktuell eindringende Wärmeenergie und $Q_ab$ die abgeführte
*Wärmeenergie ist.
