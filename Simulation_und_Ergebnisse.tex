\chapter{Simulation und Ergebnisse}
\minitoc
\label{chp:sue}
Im folgenden Kapitel werden die Ergebnisse der Simulation vorgestellt und
ausgewertet. Die Daten f\"ur prognostizierte und die tats\"achlich generierte
elektrische Leistung durch Windenergie in der Vattenfall-Regelzone (heute
50Hertz-Regelzone) f\"ur das Jahr 2005 bilden die Datengrundlage f\"ur die
Implementierung des Kooperationsalgorithmus. Der Kooperationsalgorithmus wurde
in den vorangegangenen Kapiteln vorgestellt. Die Annahmen, die die jeweiligen
K\"alteeinheiten spezifizieren, wurden von Caroline M\"oller im Rahmen ihrer
Diplomarbeit entwickelt\cite{caro}. Die Substanzmasse in den K\"uhlbereichen
wurde aus Gr\"unden der Vereinfachung als Konstant angenommen (Es wird
angenommen, dass die Kauf der Ware durch permanentes Nachliefern neutralisiert
wird). Aus diesem Grund darf die Temperatur\"anderung nur aufgrund der
zugef\"uhrten Ladeleistung erfolgen. F\"ur drei zuf\"allig ausgew\"ahlte
aufeinander folgende Tage des Jahres wird die St\"atigkeit des
Temperaturverlaufes eine K\"uhleinheit des Modellsupermarktes revidiert.
Zus\"atzlich wird der Einfluss der Kooperation auf die Ladeleistung der
Modellsupermarktkette untersucht. Zum Abschluss wird durch Variation der
Parameter in den Konfigurationsdateien das Programm auf m\"ogliche
Skalierbarkeit untersucht. Durch das objektorientierte Modell entsteht das
Potenzial, die erforderliche Speicherkapazit\"at durch unterschiedliche Ma\ss
nahmen zu erh\"ohen z.B. durch die Variation der Anzahl von Objekten der Klassen
Supermarkt und Refrigerator. Im ersten Schritt wird untersucht, wie die
Verdoppelung der Speicherkapazit\"at auf die Regelleistung den Energieverbrauch
der Supermarktkette auswirkt. Im zweiten Schritt wird untersucht, ob die Art und
Weise, wie die Speicherkapazit\"at erh\"oht wurde, einen Einfluss auf das
Verhalten des Systems nimmt. Im vorangegangenen Kapitel wurde erkl\"art, wie mit
Hilfe der Konfigurationsdateien die Anzahl der Superm\"arkte und der
K\"uhleinheiten im System ver\"andert werden kann. Aufgrund des im
Kooperationsmodell umgesetzten Verfahrens der gleichm\"a\ss igen Aufteilung der
vorhandenen Ladeleistung auf alle K\"alteeinheiten im Netz wird erwartet, dass
die Art und Weise wie man Anzahl der K\"alteeinheiten erh\"oht, keinen Einflu\ss
$ $ auf das Ergebnis der Simulation haben d\"urfen, solange die Anzahl, die
Beladung und die technische Ausf\"uhrung \"ubereinstimmen. Eine Best\"atigung
der Vermutung w\"urde bedeuten, dass die implementierte Kommunikation zwischen
Objekte nach gew\"unschter Weise funktioniert. Dadurch entsteht die Perspektive,
mit dem Programm den Einsatz der K\"altespeicher im elektrischen
Energieversorgungsnez mit weiteren Lastmanagement-Algorithmen und Ausf\"uhrungen
zu untersuchen.

\subsection{Auswertung Kooperationskonzept}
Die Simulation wird f\"ur vier F\"alle durchgef\"uhrt (vgl. \cref{t:faelle} auf
der Seite \pageref{t:faelle}). Im im Fall 1 werden die Multiplikatoren in den
Konfigurationsdateien \matref{cgrid} und \matref{csuper} so ver\"andert, dass
die Anzahl der Superm\"arkte 1500 der aktuellen realen Situation f\"ur das
Gebiet Berlin-Brandenburg stark nahe kommt \cite{caro}. Das Modellsupermarkt
enth\"alt dabei f\"unf verschiedene K\"uhleinheiten. Das Programm erzeugt mit
Hilfe dieser Konfigurationsdateien ein Objekt der Klasse Supermarkt und f\"unf
Objekte der Klasse Refrigerator. Durch die Multiplikatoren wird mit Hilfe der
sechs Objekte, das Verhalten 7500 unterschiedlicher K\"alteeinheiten mit je 1500
Einheiten simuliert (vgl. Fall 1 in der \cref{t:faelle}). Der Modellsupermarkt
hat einen angenommenen durchschnittlichen Verbrauch von von 75,1384 MWh im Jahr
\cite{caro}. Durch die Kooperation erh\"oht sich der Verbrauch auf 75,4479 MWh
um rund 0,00441 $\%$.  F\"ur das gesamte Netz mit 1500 Superm\"arkten sind das
464,25 MWh an Mehrbedarf im Jahr. Der Wert f\"ur die Regellenergie $W_{wo}$
f\"uhr das ganze Jahr 2005 \todo{Gleichung ref} ohne die Kooperation lag bei
rund 3,6 $10^6$ MWh. Durch die Kooperation konnte in der Simulation der Wert um
rund ein Prozent mit 37,458 MWh gesenkt werden. Tr\"agt der Winparkbetreiber die
die Kosten des Mehrbedarfs f\"ur den Supermarkt, dann ist durch die Einsparung
an Regelenergie die Kooperationsrendite um die 7968,5 Prozent m\"oglich.

\begin{table}
\footnotesize{
\centering
\begin{tabularx}{\textwidth}{X|X|X|X|X|X|X|X|X|}
\cline{2-9}
& \multicolumn{2}{c|}{\textbf{Fall 1}} & \multicolumn{2}{c|}{\textbf{Fall 2}}
&  \multicolumn{2}{c|}{\textbf{Fall 3}} &  \multicolumn{2}{c|}{\textbf{Fall 4}}\\
 \cline{2-9}
& Super- \linebreak markt & Refri- \linebreak gerator & Super- \linebreak markt
& Refri- \linebreak gerator & Super- \linebreak markt & Refri-\linebreak gerator
& Super- \linebreak markt & Refri-\linebreak gerator \\
\hline
\multicolumn{1}{|l|}{Multiplikator} & \multicolumn{1}{c|}{1500} &
\multicolumn{1}{c|}{1} & \multicolumn{1}{c|}{3000} & \multicolumn{1}{c|}{1} &
\multicolumn{1}{c|}{1500} & \multicolumn{1}{c|}{2} & \multicolumn{1}{c|}{1500} &
\multicolumn{1}{c|}{1} \\
\hline
\multicolumn{1}{|l|}{Objektenanzahl} & \multicolumn{1}{c|}{1} &
\multicolumn{1}{c|}{5}& \multicolumn{1}{c|}{1} & \multicolumn{1}{c|}{5} &
\multicolumn{1}{c|}{1}& \multicolumn{1}{c|}{5} & \multicolumn{1}{c|}{2}&
\multicolumn{1}{c|}{10} \\
\hline
\multicolumn{1}{|l|}{Gesamtzahl im Netz} & \multicolumn{1}{c|}{1500} &
\multicolumn{1}{c|}{7500}& \multicolumn{1}{c|}{3000} & \multicolumn{1}{c|}{15000} &
\multicolumn{1}{c|}{1500} & \multicolumn{1}{c|}{15000} & \multicolumn{1}{c|}{3000}&
\multicolumn{1}{c|}{15000} \\
\hline
\multicolumn{1}{|l|}{Gesamtzahl der Objekte} & \multicolumn{2}{c|}{6} &
\multicolumn{2}{c|}{6} & \multicolumn{2}{c|}{6} & \multicolumn{2}{c|}{12}\\
\hline
\hline
\multicolumn{1}{|l|}{$W_{r_{wo}}$ in $10^6$ MWh } & \multicolumn{2}{c|}{3,6116} &
\multicolumn{2}{c|}{} & \multicolumn{2}{c|}{11} & \multicolumn{2}{c|}{11}\\
\hline
\multicolumn{1}{|l|}{$W_{r_{w}}$ in $10^6$ MWh} & \multicolumn{2}{c|}{3,5742} &
\multicolumn{2}{c|}{5} & \multicolumn{2}{c|}{11} & \multicolumn{2}{c|}{11}\\
\hline
\multicolumn{1}{|l|}{$(W_{r_{wo}} - W_{r_{w}})$ in MWh} &
\multicolumn{2}{c|}{37.458} &
\multicolumn{2}{c|}{5} & \multicolumn{2}{c|}{11} & \multicolumn{2}{c|}{11}\\
\hline
\hline
\multicolumn{1}{|l|}{$W_{dem}$ in MWh } & \multicolumn{2}{c|}{112.710} &
\multicolumn{2}{c|}{} & \multicolumn{2}{c|}{11} & \multicolumn{2}{c|}{11}\\
\hline
\multicolumn{1}{|l|}{$W_{coop}$ in MWh} & \multicolumn{2}{c|}{113.170} &
\multicolumn{2}{c|}{5} & \multicolumn{2}{c|}{11} & \multicolumn{2}{c|}{11}\\
\hline
\multicolumn{1}{|l|}{$(W_{dem} - W_{coop})$ in MWh} & \multicolumn{2}{c|}{464,25} &
\multicolumn{2}{c|}{5} & \multicolumn{2}{c|}{11} & \multicolumn{2}{c|}{11}\\
\hline
\hline
\multicolumn{1}{|l|}{Kooperationsgewinn in \%} & \multicolumn{2}{c|}{7968,5} &
\multicolumn{2}{c|}{5} & \multicolumn{2}{c|}{11} & \multicolumn{2}{c|}{11}\\
\hline


\end{tabularx}
}
\caption{Simulation F\"alle}
\label{t:faelle}
\end{table}

In der \cref{fig:threep} wird der zeitliche Verlauf der Leistungsdaten des
Widparkbetreibers und die daraus f\"ur die Kooperation ermittelte Werte \"uber
drei Tage vom 14.1.2005 bis 16.1.2005 st\"undlich aufgetragen. Die Prognose
\"uber die Windleistung\footnote{ In der Abbildung als
\textit{gen\_output\_DA} bezeichnet.} am ersten Tag von Mitternacht und bis zum
Mittag war relativ konstant zwischen 3000 MW und 4000 MW angegeben. Daraus wurde
die Leistung $P_{net}$\footnote{ In der Abbildung als \textit{power\_net}
bezeichnet} (vgl.  \cref{eq:pnet} auf der Seite \pageref{eq:pnet}), die beim
Netzbetreiber aufgrund des Kooperationskonzepts anzumelden ist, berechnet. Die
tats\"achlich vom Windpark generierte Leistung\footnote{ In der Abbildung
als \textit{gen\_output\_DA} bezeichnet.} liegt zeitweise mit rund 2500 MW
unter der Prognose. Die Leistung, mit der der Windpark sich beim Supermarkt und
beim Netzbetreiber auf Grund der hohen Prognosewerte verpflichtet, kann nicht
geliefert werden. In der \cref{fig:threet} auf der Seite \cref{fig:threet} ist
im dritten Nebenbild zu erkennen, dass der Windparkbetreiber sich beim
Supermarkt mit der Lieferung des kompletten durchschnittlichen st\"undlichen
Bedarfs\footnote{ In der Abbildung als \textit{demand} bezeichnet.}
verpflichtet\footnote{ In der Abbildung wird diese Leistung als
\textit{power promised} bezeichnet (vgl. die \cref{eq:psm} auf der Seite
\pageref{eq:psm}).}  hat. Infolge dessen hat sich der Supermarkt beim
Netzbetreiber komplett als Last abgemeldet\footnote{ In der Abbildung als
\textit{firm commitment} bezeichnet}. Im K\"uhlbereich darf

% \Ergebnisse
\begin{figure}[h] \begin{center}
\includegraphics[scale=0.40]{images/Simulation/three_days_power} \end{center}
\vspace{-25pt} \caption{Leistungswerte Windparkbetreiber f\"ur die Tage
14.1.05-16.1.05} \label{fig:threep} \end{figure}

\begin{figure}[h]
\begin{center}
\includegraphics[scale=0.40]{images/Simulation/three_days2}
\end{center}
\vspace{-25pt}
\caption{Temperatur und Leistungswerte f\"ur die Tage 14.1.05-16.1.05}
\label{fig:threet}
\end{figure}


\begin{figure}[h]
	\begin{center}
	\includegraphics[scale=0.39]{images/Simulation/415}
	\end{center}
\caption{Ausgabe plotCooling.m}
\label{fig:ausgabe}
\end{figure}


