\chapter{Simulation und Ergebnisse}
\minitoc
\label{chp:sue}
Im folgenden Kapitel werden die Auswertung der Simulation statt. Die Auswirkung
des im vorangegangenen Kapitel erkl\"arten Kooperationskonzeptes zwischen einem
Windparkbetreiber und einer Supermarketkette wird auf der Basis der
Windleistungsdaten der Vattenfall-Regelzone (heute 50Hertz-Regelzone) f\"ur das
Jahr 2005 untersucht. F\"ur drei zuf\"allig ausgew\"ahlte aufeinander folgende
Tage des Jahres wird die St\"atigkeit des Temperaturverlaufes eine K\"uhleinheit
des Modellsupermarktes revidiert. Zus\"atzlich wird der Einfluss der
Kooperation auf die Ladeleistung der Modellsupermarktkette untersucht. Zum
Abschluss wird durch Variation der Parameter in den Konfigurationsdateien das
Programm auf m\"ogliche Skalierbarkeit untersucht. Durch das objektorientierte
Modell entsteht das Potenzial, die erforderliche Speicherkapazit\"at durch
unterschiedliche Ma\ss nahmen zu erh\"ohen. Der Gegenstand der Untersuchung wird
die Variation der Anzahl unterschiedlicher Objekte des Modells, sodass die
doppelte Speicherkapazit\"at erreicht wird. Aufgrund des im Kooperationsmodell
umgesetzten Verfahrens der gleichm\"a\ss igen Aufteilung der vorhandenen
Ladeleistung auf alle K\"alteeinheiten im Netz wird erwartet, dass die
Ver\"anderungen keinen Einflu\ss $ $ auf das Ergebnis der Simulation haben,
solange die Anzahl, die Beladung und die technische Ausf\"uhrung
\"ubereinstimmen. Eine Best\"atigung der Vermutung w\"urde bedeuten, dass die
implementierte Kommunikation zwischen Objekte funktioniert. Dadurch entsteht die
Perspektive, mit dem Programm den Einsatz der K\"altespeicher im elektrischen
Energieversorgungsnez mit weiteren Lastmanagement-Algorithmen und Ausf\"uhrungen
zu untersuchen.

der Einfluss der Bandbreite der Leistung $P_{ms}$ (vgl.
\cref{eq:psm} auf der Seite \pageref{eq:psm}), die der Windparkbetreiber dem
Supermarkt einen Tag voraus verspricht, auf die Menge der Regelleistung
untersucht. Die Menge der Leistung $P_{ms}$ ist von der Genauigkeit Prognose und
der Menge der prognostizierten Leistung abh\"angig.



\begin{figure}[h]
	\begin{center}
	\includegraphics[scale=0.39]{images/Simulation/415}
	\end{center}
\caption{Ausgabe plotCooling.m}
\label{fig:ausgabe}
\end{figure}

\begin{table}
\footnotesize
\centering
\begin{tabularx}{\textwidth}{X|X|X|X|X|X|X|X|X|}
\cline{2-9}
& \multicolumn{2}{c|}{\textbf{Fall 1}} & \multicolumn{2}{c|}{\textbf{Fall 2}}
&  \multicolumn{2}{c|}{\textbf{Fall 3}} &  \multicolumn{2}{c|}{\textbf{Fall 4}}\\
 \cline{2-9}
& Super- \linebreak markt & Refri- \linebreak gerator & Super- \linebreak markt
& Refri- \linebreak gerator & Super- \linebreak markt & Refri-\linebreak gerator
& Super- \linebreak markt & Refri-\linebreak gerator \\
\hline
\multicolumn{1}{|l|}{Multiplikator} & \multicolumn{1}{c|}{1500} &
\multicolumn{1}{c|}{1} & \multicolumn{1}{c|}{3000} & \multicolumn{1}{c|}{1} &
\multicolumn{1}{c|}{1500} & \multicolumn{1}{c|}{2} & \multicolumn{1}{c|}{1500} &
\multicolumn{1}{c|}{1} \\
\hline
\multicolumn{1}{|l|}{Objektenanzahl} & \multicolumn{1}{c|}{1} &
\multicolumn{1}{c|}{5}& \multicolumn{1}{c|}{1} & \multicolumn{1}{c|}{5} &
\multicolumn{1}{c|}{1}& \multicolumn{1}{c|}{10} & \multicolumn{1}{c|}{1}&
\multicolumn{1}{c|}{10} \\
\hline
\multicolumn{1}{|l|}{Gesamtzahl im Netz} & \multicolumn{1}{c|}{1500} &
\multicolumn{1}{c|}{7500}& \multicolumn{1}{c|}{3000} & \multicolumn{1}{c|}{15000} &
\multicolumn{1}{c|}{1500} & \multicolumn{1}{c|}{15000} & \multicolumn{1}{c|}{2}&
\multicolumn{1}{c|}{10} \\
\hline
\multicolumn{1}{|l|}{Gesamtzahl der Objekte} & \multicolumn{2}{c|}{6} &
\multicolumn{2}{c|}{5} & \multicolumn{2}{c|}{11} & \multicolumn{2}{c|}{11}\\
\hline
\end{tabularx}
\caption{Simulation F\"alle}
\end{table}
