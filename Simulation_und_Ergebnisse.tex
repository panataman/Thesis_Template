\chapter{Simulation und Ergebnisse}
\minitoc
\label{chp:sue}
Im folgenden Kapitel werden die Ergebnisse der Simulation vorgestellt und
ausgewertet. Die Daten f\"ur prognostizierte und die tats\"achlich generierte
elektrische Leistung durch Windenergie in der Vattenfall-Regelzone (heute
50Hertz-Regelzone) f\"ur das Jahr 2005 bilden die Datengrundlage f\"ur die
Implementierung des Kooperationsalgorithmus. Der Kooperationsalgorithmus wurde
in den vorangegangenen Kapiteln vorgestellt. Die Annahmen, die die jeweiligen
K\"alteeinheiten spezifizieren, wurden von Caroline M\"oller im Rahmen ihrer
Diplomarbeit entwickelt\cite{caro}. Die Substanzmasse in den K\"uhlbereichen
wurde aus Gr\"unden der Vereinfachung als Konstant angenommen (Es wird
angenommen, dass die Kauf der Ware durch permanentes Nachliefern neutralisiert
wird). Aus diesem Grund darf die Temperatur\"anderung nur aufgrund der
zugef\"uhrten Ladeleistung erfolgen. F\"ur drei zuf\"allig ausgew\"ahlte
aufeinander folgende Tage des Jahres wird die St\"atigkeit des
Temperaturverlaufes eine K\"uhleinheit des Modellsupermarktes revidiert.
Zus\"atzlich wird der Einfluss der Kooperation auf die Ladeleistung der
Modellsupermarktkette untersucht. Zum Abschluss wird durch Variation der
Parameter in den Konfigurationsdateien das Programm auf m\"ogliche
Skalierbarkeit untersucht. Durch das objektorientierte Modell entsteht das
Potenzial, die erforderliche Speicherkapazit\"at durch unterschiedliche Ma\ss
nahmen zu erh\"ohen z.B. durch die Variation der Anzahl von Objekten der Klassen
Supermarkt und Refrigerator. Im ersten Schritt wird untersucht, wie die
Verdoppelung der Speicherkapazit\"at auf die Regelleistung den Energieverbrauch
der Supermarktkette auswirkt. Im zweiten Schritt wird \"uberpr\"uft, ob die Art
und Weise, wie die Speicherkapazit\"at erh\"oht wurde, einen Einfluss auf das
Verhalten des Systems nimmt. Im vorangegangenen Kapitel wurde erkl\"art, wie mit
Hilfe der Konfigurationsdateien die Anzahl der Superm\"arkte und der
K\"uhleinheiten im System ver\"andert werden kann. Aufgrund des im
Kooperationsmodell umgesetzten Verfahrens der gleichm\"a\ss igen Aufteilung der
vorhandenen Ladeleistung auf alle K\"alteeinheiten im Netz wird erwartet, dass
die Art und Weise wie man Anzahl der K\"alteeinheiten erh\"oht, keinen Einflu\ss
$ $ auf das Ergebnis der Simulation haben d\"urfen, solange die Anzahl, die
Beladung und die technische Ausf\"uhrung \"ubereinstimmen. Eine Best\"atigung
der Vermutung w\"urde bedeuten, dass die implementierte Kommunikation zwischen
Objekte nach gew\"unschter Weise funktioniert. Dadurch entsteht die Perspektive,
mit dem Programm den Einsatz der K\"altespeicher im elektrischen
Energieversorgungsnez mit weiteren Lastmanagement-Algorithmen und Ausf\"uhrungen
zu untersuchen.

\section{Auswertung}
Die Simulation wird f\"ur vier F\"alle durchgef\"uhrt (vgl. \cref{t:faelle} auf
der Seite \pageref{t:faelle}). Im im Fall 1 werden die Multiplikatoren in den
Konfigurationsdateien \matref{cgrid} und \matref{csuper} so ver\"andert, dass
die Anzahl der Superm\"arkte 1500 der aktuellen realen Situation f\"ur das
Gebiet Berlin-Brandenburg stark nahe kommt \cite{caro}. Das Modellsupermarkt
enth\"alt dabei f\"unf verschiedene K\"uhleinheiten. Das Programm erzeugt mit
Hilfe dieser Konfigurationsdateien ein Objekt der Klasse Supermarkt und f\"unf
Objekte der Klasse Refrigerator. Durch die Multiplikatoren wird mit Hilfe der
sechs Objekte, das Verhalten 7500 unterschiedlicher K\"alteeinheiten mit je 1500
Einheiten simuliert (vgl. Fall 1 in der \cref{t:faelle}). Der Modellsupermarkt
hat einen angenommenen durchschnittlichen Verbrauch von von 75,1384 MWh im Jahr
\cite{caro}. Durch die Kooperation erh\"oht sich der Verbrauch auf 75,4479 MWh
um rund 0,00441 $\%$.  F\"ur das gesamte Netz mit 1500 Superm\"arkten sind das
464,25 MWh an Mehrbedarf im Jahr. Der Wert f\"ur die Regellenergie $W_{wo}$
f\"uhr das ganze Jahr 2005 (vgl. die \cref{eq:wrwo} auf der Seite
\pageref{eq:wrwo}) ohne die Kooperation lag bei rund 3,6 $10^6$ MWh. Durch die
Kooperation konnte in der Simulation der Wert um rund ein Prozent mit 37,458 MWh
gesenkt werden. Tr\"agt der Winparkbetreiber die Kosten des Mehrbedarfs
f\"ur den Supermarkt, dann ist durch die Einsparung an Regelenergie die
Kooperationsrendite um die 7968,5 Prozent m\"oglich (vgl. \cref{t:faelle}).
F\"ur die Verdoppelung der Anzahl der K\"altespeicher ist das Ergebnis der
Simulation auch eine Verdoppelung der Einsparung an Regelleistung. Eine negative
Folge ist die Verdoppelung des Mehrbedarfs f\"ur den Supermarkt. Dadurch ist
keine \"Anderung der Kooperationsrendite beobachtet worden.\\

Zum besseren Verst\"andnis, welche Dateien und wie zur Simulation jeweiligen
F\"alle ver\"adert wurden sind sie im Anhang zus\"atzlich abgebildet.
Die Konfigurationsdatein, mit denen Fall 1 simuliert wurden, sind in der
\matref{cgrid} und \matref{csuper} zu sehen.
Die Konfigurationsdatein, mit denen Fall 2 simuliert wurden, sind in der
\matref{dsf} und \matref{csuper} zu sehen.
Die Konfigurationsdatein, mit denen Fall 3 simuliert wurden, sind in der
\matref{cgrid} und \matref{dff} zu sehen.
Die Konfigurationsdatein, mit denen Fall 4 simuliert wurden, sind in der
\matref{fall} und \matref{csuper} zu sehen.

Aus der \cref{t:faelle} ist ersichtlich, dass die Vergr\"osserung der
Speicherkapazit\"at durch die Konfigurationsdateien funktioniert. F\"ur beliebig
gro\ss e Energieversorgungsnetze k\"onnen dadurch Simulationen durchgef\"uhr
werden.

% \include{tabelle}
\begin{table}
\footnotesize{
\centering
\begin{tabularx}{\textwidth}{X|X|X|X|X|X|X|X|X|}
\cline{2-9}
& \multicolumn{2}{c|}{\textbf{Fall 1}} & \multicolumn{2}{c|}{\textbf{Fall 2}}
&  \multicolumn{2}{c|}{\textbf{Fall 3}} &  \multicolumn{2}{c|}{\textbf{Fall 4}}\\
 \cline{2-9}
& Super- \linebreak markt & Refri- \linebreak gerator & Super- \linebreak markt
& Refri- \linebreak gerator & Super- \linebreak markt & Refri-\linebreak gerator
& Super- \linebreak markt & Refri-\linebreak gerator \\
\hline
\multicolumn{1}{|l|}{Multiplikator} & \multicolumn{1}{c|}{1500} &
\multicolumn{1}{c|}{1} & \multicolumn{1}{c|}{3000} & \multicolumn{1}{c|}{1} &
\multicolumn{1}{c|}{1500} & \multicolumn{1}{c|}{2} & \multicolumn{1}{c|}{1500} &
\multicolumn{1}{c|}{1} \\
\hline
\multicolumn{1}{|l|}{Objektenanzahl} & \multicolumn{1}{c|}{1} &
\multicolumn{1}{c|}{5}& \multicolumn{1}{c|}{1} & \multicolumn{1}{c|}{5} &
\multicolumn{1}{c|}{1}& \multicolumn{1}{c|}{5} & \multicolumn{1}{c|}{2}&
\multicolumn{1}{c|}{10} \\
\hline
\multicolumn{1}{|l|}{Gesamtzahl im Netz} & \multicolumn{1}{c|}{1500} &
\multicolumn{1}{c|}{7500}& \multicolumn{1}{c|}{3000} & \multicolumn{1}{c|}{15000} &
\multicolumn{1}{c|}{1500} & \multicolumn{1}{c|}{15000} & \multicolumn{1}{c|}{3000}&
\multicolumn{1}{c|}{15000} \\
\hline
\multicolumn{1}{|l|}{Gesamtzahl der Objekte} & \multicolumn{2}{c|}{6} &
\multicolumn{2}{c|}{6} & \multicolumn{2}{c|}{6} & \multicolumn{2}{c|}{12}\\
\hline
\hline
\multicolumn{1}{|l|}{$W_{r_{wo}}$ in $10^6$ MWh } & \multicolumn{8}{c|}{3,6116}\\
\hline
\multicolumn{1}{|l|}{$W_{r_{w}}$ in $10^6$ MWh} & \multicolumn{2}{c|}{3,5742} &
\multicolumn{2}{c|}{3,5381} &
\multicolumn{2}{c|}{3,5381} &
\multicolumn{2}{c|}{3,5381}\\
\hline
\multicolumn{1}{|l|}{$|W_{r_{wo}} - W_{r_{w}}|$ in MWh} &
\multicolumn{2}{c|}{37.458} &
\multicolumn{2}{c|}{73.530} &
\multicolumn{2}{c|}{73.649} &
\multicolumn{2}{c|}{73.530}\\
\hline
\hline
\multicolumn{1}{|l|}{$W_{dem}$ in MWh } & \multicolumn{2}{c|}{112.707,62} &
\multicolumn{2}{c|}{225.415,23} &
\multicolumn{2}{c|}{225.415,23} &
\multicolumn{2}{c|}{225.415,23}\\
\hline
\multicolumn{1}{|l|}{$W_{coop}$ in MWh} & \multicolumn{2}{c|}{113.171,86} &
\multicolumn{2}{c|}{226.353,68} &
\multicolumn{2}{c|}{226.355,31} &
\multicolumn{2}{c|}{226.353,68}\\
\hline
\multicolumn{1}{|l|}{$(W_{dem} - W_{coop})$ in MWh} & \multicolumn{2}{c|}{464,24} &
\multicolumn{2}{c|}{938.450} &
\multicolumn{2}{c|}{940,080} &
\multicolumn{2}{c|}{938.450}\\
\hline
\hline
\multicolumn{1}{|l|}{Kooperationsgewinn in \%} & \multicolumn{2}{c|}{7968,5} &
\multicolumn{2}{c|}{7935,3} &
\multicolumn{2}{c|}{7934,3} &
\multicolumn{2}{c|}{7935,3}\\
\hline
\end{tabularx}
}
\caption{Simulation F\"alle}
\label{t:faelle}
\end{table}

Die Regelenergie berechnete sich dadurch, dass der Summe die Stundenwerte f\"ur
die vom Windparkbetreiber tats\"achlich generierte Leistungswerte \"uber
das ganze Jahr die Summe der Prognosestundenwerte \"uber das Ganze Jahr
abgezogen. Anschlie\"ss end wird daraus der Betrag genommen.
\begin{equation}
	W_{r_{wo}} = |\sum^{365}_{d=1}\sum^{24}_{h=1}(P_{RT\,dh}-P_{DA\,dh})|
\label{eq:wrwo}
\end{equation}
\begin{description}[\dth]
\item[$W_{r_{wo}}$] Regelenergie ohne Kooperation zum Ausgleich der
Prognosefehler.
\item[$P_{RT}$] Windleistungsdaten. Windleistung, die tats\"achlich vom
Windpark generiert wurde in MW
\item[$P_{DA}$] Vorhersage der Windleistung (day ahead) in MW
\item[$d$] Laufvariable f\"ur die Tage
\item[$h$] Laufvariable f\"ur die Stunden
\end{description}

Die durch die Kooperation entstanden Regelenergie wird berechnet, indem zuerst
der Betrag aus der Differenz zwischen der tats\"achlich vom Windpark generierten
Leistung summiert \"uber das Jahr die auch \"uber das Jahr summierte Leistung
abgezogen wird, mit der der Windparkbetreiber sich beim Netzbetreiber angemeldet
hat. Konnte der Supermarkt Winleistung integrieren, ohne zus\"atzliche Leistung
aus dem Netz zu beziehen so muss der Betrag um diese auch \"uber das ganze Jahr
summierte Leistung abziehen. Wurde vom Supermarkt zus\"atzlich Leistung aus dem
Netz bezogen, so wird die restliche Regelenergie um die \"uber das ganze Jahr
summierte zus\"atzliche Regelleistung erh\"oht.

\begin{equation}
W_{r_{w}} = |\sum^{365}_{d=1}\sum^{24}_{h=1}(P_{RT_{dh}}-P_{net_{dh}})| -
\sum^{365}_{d=1}\sum^{24}_{h=1}(P_{involved_{dh}} - P_{additionally_{dh}})
\label{eq:wrw}
\end{equation}

\begin{description}[\dth]
\item[$W_{r_{wo}}$] Regelenergie ohne Kooperation zum Ausgleich der
Prognosefehler.
\item[$P_{RT}$] Windleistungsdaten. Windleistung, die tats\"achlich vom
Windpark generiert wurde in MW
\item[$P_{DA}$] Vorhersage der Windleistung (day ahead) in MW
\item[$d$] Laufvariable f\"ur die Tage
\item[$h$] Laufvariable f\"ur die Stunden
\end{description}

Der Betrag der Differenz |$W_{r_{wo}}- W_{r_{w}}$| bildet den Wert der
eingesparten Regelenergie.

Der durchschnittliche Jahresverbrauch f\"ur alle Superm\"arkte $W_{dem}$ ist die
Summe des durchschnittlichen st\"undlichen Bedarfs $P_{dem}$ f\"ur alle Stunden
und Tage des Jahres.

\begin{equation}
W_{dem} = \sum^{365}_{d=1}\sum^{24}_{h=1}P_{dem_{dh}}
\label{eq:wdem}
\end{equation}

\begin{description}[\dth]
\item[$W_{dem}$] Der durchschnittliche Jahresverbrauch f\"ur die Supermaktkette
\item[$P_{dem}$] St\"undlicher Leistungsbedarf des Supermarkts (demand)
\end{description}

Der durchschnittliche Jahresverbrauch f\"ur alle Superm\"arkte $W_{dem}$ ist die
Summe des durchschnittlichen st\"undlichen Bedarfs $P_{dem}$ f\"ur alle Stunden
und Tage des Jahres.

\begin{equation}
W_{coop} = \sum^{365}_{d=1}\sum^{24}_{h=1}P_{v_{dh}}
\label{eq:wcoop}
\end{equation}

\begin{description}[\dth]
\item[$W_{coop}$] Der Jahresverbrauch f\"ur die Supermaktkette infolge des
Kooperation
\item[$P_{v}$] Der st\"undliche Leistungsverbrauch des Supermarkts aufgrund der
Kooperation
\end{description}

Die Differenz $W_{dem}- W_{coop}$| bildet den Wert des Jahresmehrbedarfs an
Energie aufgrund der Kooperation.

In der \cref{fig:threep} wird der zeitliche Verlauf der Leistungsdaten des
Widparkbetreibers und die daraus f\"ur die Kooperation ermittelte Werte \"uber
drei Tage vom 14.1.2005 bis 16.1.2005 st\"undlich aufgetragen. Die Prognose
\"uber die Windleistung\footnote{ In der Abbildung als \textit{gen\_output\_DA}
bezeichnet.} am ersten Tag von Mitternacht und bis zum Mittag war relativ
konstant zwischen 3000 MW und 4000 MW angegeben. Daraus wurde die Leistung
$P_{net}$\footnote{ In der Abbildung als \textit{power\_net} bezeichnet} (vgl.
\cref{eq:pnet} auf der Seite \pageref{eq:pnet}), die beim Netzbetreiber aufgrund
des Kooperationskonzepts anzumelden ist, berechnet. Die tats\"achlich vom
Windpark generierte Leistung\footnote{ In der Abbildung als
\textit{gen\_output\_DA} bezeichnet.} liegt zeitweise mit rund 2500 MW unter der
Prognose. Die Leistung, mit der der Windpark sich beim Supermarkt und beim
Netzbetreiber auf Grund der hohen Prognosewerte verpflichtet, kann nicht
geliefert werden. In der \cref{fig:threet} auf der Seite \cref{fig:threet} ist
im dritten Subplot zu erkennen, dass der Windparkbetreiber sich beim Supermarkt
mit der Lieferung des kompletten durchschnittlichen st\"undlichen
Bedarfs\footnote{ In der Abbildung als \textit{demand} bezeichnet.}
verpflichtet\footnote{ In der Abbildung wird diese Leistung als \textit{power
promised} bezeichnet (vgl. die \cref{eq:psm} auf der Seite \pageref{eq:psm}).}
hat. Infolge dessen hat sich der Supermarkt beim Netzbetreiber komplett als Last
abgemeldet\footnote{ In der Abbildung als \textit{firm commitment} bezeichnet}.
Im K\"uhlbereich darf die Temperatur nicht \"uber die obere Grenze steigen. Zum
Einhalten dieser Bedingung muss der Supermarkt zus\"atzlich Leistung\footnote{
In der Abbildung mit \textit{additionally power bezeichnet} bezeichnet.}
aus dem Netz ziehen. F\"ur den Netzbetreiber Verursacht diese unangemeldete Last
einen zus\"atzlichen Regelaufwand. Der Verlauf dieser Energie ist im
Subplot in der \cref{fig:threet} enthalten. Es ist zu erkennen, dass der Bedarf
des Supermarkts\footnote{ In der Abbildung als \textit{charge} bezeichnet.} an
diesem Tag komplett durch die zus\"atzliche Leistung aus dem Netz gedeckt wird.
Die Temperatur in den K\"uhlbereichen tiefgek\"uhlte Tiefk\"uhltruhe
steckerfertig und Tiefk\"uhl K\"uhlzellen bleibt in diesem Zeitraum bei der
maximal zul\"assigen Temperatur von -18 $\grad$ C. In den Stunden 27 bis 30
liegen die Werte f\"ur tats\"achlich generierte Leistun \"uber der Prognose. In
diesem Zeitbereich ist auch ein deutlicher Anstieg der Ladeleistung des
Supermarkts zu beobachten. Die Temperatur wurde dadurch gesenkt.  Es wurde keine
zus\"atzliche Leistung aus dem Netz beansprucht.  Die Aufladung wurde komplett
durch die Windleistung vorgenommen\footnote{ In der Abbildung wird die Leistung,
die ausschlie\ss lich vom Windpark beansprucht wird, als \textbf{wind only}
bezeichnet}. Die Prognose wurde der realen Werte f\"ur den neuen Tag angepasst.
Der Tag 15.1.2005 zeichnet sich durch eine deutliche Windschw\"ache aus, sodass
der Supermarkt mit dem gr\"o\ss ten Teil seines Bedarfs sich nicht beim
Netzbetreiber abgemeldet hat. Der Windpark konnte aber auch die Differenz nicht
bedienen, sodass nur die aus dem Netz bestellte Leistung verbraucht werden
konnte. Diese Menge ist jedoch kleiner als der durchschnittliche st\"undliche
Bedarf.  Aus diesem Grund steigt die Temperatur in den K\"uhlbereichen.  In den
Stunden zwischen 36 und 40 ist die bestellte Leistung so gro\ss ,$ $ dass ein
Teil der K\"uhlzellen damit versorgt wird, und die Temperatur konstant gehalten
werden kann. In den Stunden 43 bis 60 liegt die tats\"achliche generierte
Windleistung deutlich \"uber der Prognose, sodass der Supermarkt komplett von
der Windversorgung gespeist wurde. Die Speicherkapazit\"at wurde vollst\"andig
ausgenutzt und die Temperatur auf das Minimun heruntergek\"uhlt. Aus diesem
Grund konnte der Supermarkt in der Stunde 60, als wieder keine Leistung
geliefert werden konnte, vollst\"andig auf Leistung verzichten. In den folgenden
Stunden wird aufgrund der Temperaurbeschr\"ankungen die zus\"atzliche Leistung
aus dem Netz beansprucht, bis wieder Mehrangebot an Windleistung gibt. Die
Temperatur bleibt stehts in den definierten Grenzen und reagiert gem\"a\ss $ $
den Erwartungen auf die Netzsituation. Die vollst\"andige \"Ubersicht \"uber
alle K\"uhlzellen des Modellsupermarktes sowie \"uber die Leistungsferl\"aufe im
gesamten Netz f\"ur diese Tage kann aus der \cref{fig:14}, der \cref{fig:15} und
der \cref{fig:16} bezogen werden. Der Unterschiede im Temperaturverlauf sind
dadurch zu begr\"unden, dass einzelnen K\"uhleinheiten sich durch die Beladung,
die Abdichtung gegen K\"alteenergieverlust etc. unterscheiden.

\begin{figure}[h] \begin{center}
\includegraphics[scale=0.40]{images/Simulation/three_days_power} \end{center}
\vspace{-25pt} \caption{Leistungswerte Windparkbetreiber f\"ur die Tage
14.1.05-16.1.05} \label{fig:threep} \end{figure}

\begin{figure}[h]
\begin{center}
\includegraphics[scale=0.40]{images/Simulation/three_days_2}
\end{center}
\vspace{-25pt}
\caption{Temperatur und Leistungswerte f\"ur die Tage 14.1.05-16.1.05}
\label{fig:threet}
\end{figure}

\subsection*{Aussichten}

Es gib viele M\"oglichgeiten, wie das Programm verbessert werden kann. An erster
Stelle k\"onnen zur realit\"atsgetreuen Darstellung der Abl\"aufe und Verfahren
im K\"altetechnikbetrieb die Erstellung der Spezifikationsvariablen auf der
Basis von messtechnischen Verfahren zu verwirklichen. M\"ogliche Schwachstellen
des Modells k\"onnten dadurch korrigiert werden. Weitere Klassen zur Simulation
weiterer verschiedener K\"alteverlustquellen k\"onnen implementiert werden. Der
Einfluss der Vereisung des und die sinkende Leistungszahl des Verdampfers sowie
die Beschr\"ankung der Leistungsaufnahme durch technische Grenzen wurden nicht
betrachtet. Das Kaufverhalten, also der Verlust an Substanzmasse zur Speicherung
der K\"alteenergie und der Einfluss der Jahreszeiten w\"urden einen genaueren
Abbild der Realit\"at darstellen. Das Programm k\"onnte einen Grundstein f\"ur
weitere Studien auf diesem Gebiet bilden.

