\chapter{Aufgabenstellung}
\label{chap:aufst}

%%%%%%%%%%%%%%%%%%%%%%%%%%%%%%%%%%%%%%%%%%%%%%%%%%%%%%%
%%%%%%%%%%%%%%%%%%%%%%%%%%%%%%%%%%%%%%%%%%%%%%%%%%%%%%%
%%%%%%%%%%%%%%%%%%%%%%%%%%%%%%%%%%%%%%%%%%%%%%%%%%%%%%%

Die Intention der Arbeit, ist der objektorientierte Entwurf eines
\matlab-Programms zur Simulation des variablen Lastverhaltens von Kältelasten
mit Kältespeichern im Energieversorgungsnetz. 

Es sollen ausschlie\ss lich K\"alteanlagen modelliert werden, die in den
Superm\"arkten zum Einsatz kommen. Das Ergebnis der Simulation soll der
Verbrauch einer variabel gef\"uhrten Supermarktkette auf der Basis der einzelnen
in den Superm\"arkten eingesetzten K\"alteanlagen sein.

Verbrauchswerte in einem Lastflussoptimierungsprogramm zu verwenden.

\section*{Explizite Anforderungen}

Folgende explizite Anforderungen sind an das Programm gestellt:

\begin{itemize}
	\item Die Speicherung der f\"ur das Beschreiben der Modelle
	erforderlichen Parameter findet in einer Konfigurationsdatei statt.
	\item Der elektrische Energieverbrauch wird auf Grund der
	Modellparameter der Kältelasten sowie des gefahrenen Einsatzes der
	Kältelasten berechnet.
	\item Die Topologie des elektrischen Netzwerkes werden ber\"ucksichtigt.
	\item Die M\"oglichkeit einer eindeutigen Zuweisung der Verbrauchswerte
	f\"ur folgende Verursacher wird realisiert:
	\begin{itemize}
		\item Daten f\"ur jede K\"alteanlage werden einzeln gespeichert,
		wenn der Verbrauch mehrerer Anlagen simuliert wird.
		\item Daten f\"ur jede Supermarktkette werden einzeln
		gespeichert, wenn der Verbrach mehrerer Supermarktketten
		simuliert wird.
		\item Erfolgt die Simulation des Energieverbrauchs f\"ur ein
		Mehrknotennetz, so werden die Daten f\"ur jeden Knoten, einzeln
		gespeichert.
	\end{itemize}
\end{itemize}


